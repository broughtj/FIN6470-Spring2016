\documentclass[12pt]{article}

\setlength{\parindent}{0in}

\usepackage[margin=1.0in]{geometry}
\usepackage{graphicx}
\usepackage{amsmath}
\usepackage{mathrsfs}
\usepackage{textcomp}
\usepackage{setspace}
\usepackage{array,multirow}
\usepackage{tikz}
\usepackage{url}

\begin{document}

Finance 6470: Derivatives Markets           \\
Spring Semester 2021                        \\
Midterm Take-Home Exam                      \\
Due Date: April 24, 2021 at 12:00 Midnight  \\

\vspace{15mm}

\section*{Part I}

This exam is a case study on the company Metallgesellschaft Refining and Marketing (MGRM), which was an American subsidiary of a
German conglomerate. MGRM was an energy commodity merchant who employed a very interesting hedging strategy to develop innovative
long-term contracting structures for heating oil and natural gas deliveries. In 1993 the firm suffered losses totalling nearly 
$\$1.3$ billion. In the aftermath of their massive default there were many academic studies that tried to uncover the reason for their 
historic failure. In this midterm you are asked to examine some of that evidence and to employ the tools of arbitrage reasoning that we
have developed in the course to sort out the issues.

\vspace{10mm}

Please start by reading the paper \textit{Metallgesellschaft and the Economics of Synthetic Storage} by Culp and Miller. Culp and Miller
take a contrarian stance relative to the vast majority of academic studies focusing on MGRM. Most academic studies were highly critical of
MGRM. There were many facets to the saga, but there were two factors that stood out as major points on which MGRM were roundly criticized: 
1) they used a stack-and-roll hedging strategy with a 1-for-1 hedge ratio, and 2) they mismanaged their credit worthiness and suffered
financial distress. As you will see, Culp and Miller defend their hedging strategy as fundamentally sound and note that it would be nearly
impossible to see MGRM's failure as a result of financial distress. 

\vspace{10mm}

Next, please read the paper \textit{Maturity Structure of a Hedge Matters: Lessons from the Metallgesellschaft Debacle} by Mello and Parsons.
Mello and Parsons criticize MGRM for not properly hedging with a minimum-variance hedging strategy, and especially for not matching the maturity
structure of the physical deliverable leg of their strategy with their financial hedge in futures contracts. That is, for not using a strip 
hedging strategy in place of the stack-and-roll strategy that they did employ. In other words, MGRM held long-term obligations, but only short-term
hedging instruments and they were simply mismatched.

\vspace{10mm}

Culp and Miller give a rebuttal in their paper \textit{Hedging in the Theory of Corporate Finance: A Reply to Our Critics}. In their reply, they
cite the work of the economist Holbrook Working who was a pioneering scholar in futures market research. Together with Culp and Miller's reply, 
please also read the paper \textit{Futures Trading and Hedging} as well as the paper \textit{New Concepts Concerning Futures Markets and Prices}
by Working as background to this debate. 

\vspace{10mm}

When you have completed the above readings and made appropriate notes, please answer the following questions.

\vspace{10mm}

\textbf{Question 1.} Culp \& Miller referred to MGMR's hedging as \textbf{\textit{synthetic storage}}. Please outline the basics of this
		type of hedging strategy, and explain how it differs from risk minimizing hedging. In your explanation, you will want to refer to the concepts
		outlined by Holbrook Working regarding the possible different motives for hedging and how that can affect the hedging strategy that is used. 

\vspace{10mm}

\textbf{Question 2.} Why did MGRM employ a dynamic stack-and-roll hedging strategy rather than a more static strip hedge? What were the main factors
       that determined the structure of MGRM's hedging strategy? 

\vspace{10mm}

\textbf{Question 3.} Culp \& Miller essentially see MGRM as an entrepreneurial market-making firm engaged in dynamic creative arbitrage. Outline a case
       to support this contention using the scholarship that we have examined from F.A. Hayek, James Buchanan and Frank Knight. You may also use any of
	   the references that we have used so far in the course such as the historical work of Jeffrey Williams or our textbook. Feel free, of course, also
	   to cite your own novel resources. 


\vspace{10mm}

You may also find the brief review paper titled \textit{Hedging a Flow of Commodity Deliveries with Futures: Lessons from Metallgesellschaft} also by
Culp and Miller.

\newpage

\vspace{15mm}

\section*{Part II}

Next, please read the paper \textit{Metallgesellschaft: A Prudent Hedger Ruined, or a Wildcatter on NYMEX?} by Craig Pirrong. Similar to Mello and Parsons,
Pirrong criticizes MGRM's hedging strategy for deviating from the risk-minimizing standard hedge of neoclassical financial theory. Pirrong concedes that
there was a necessity for a dynamic trading strategy (why did he do that?), but suggests that MGRM was actually increasing their risk exposure with their
1-for-1 hedge ratio between spot and futures. Pirrong develops a very technically sophisticated econometric model called \textbf{\textit{backwardation-adjusted GARCH}}
to statistically estimate a time-varying minimum-variance hedge ratio. He finds that MGRM were actually massively over-hedged and by doing so were loading up on 
additional risk instead of mitigating it. You will be asked to assess the empirical findings of Pirrong and how they contribute to the debate.

\vspace{8mm}

\textbf{Note:} I do expect you to understand the basic reasoning of Pirrong's model, but not the nitty-gritty technical details. Notes will be given in class 
and as supporting material here to facilitate this understanding. It would be very easy to get overwhelmed with technical detail when reading Pirrong. Instead,
read the paper with the idea of capturing the salient features of his argument while leaving aside the mathematical technicalities. However, if you are inclined
feel free to engage the mathematical and econometric details if you feel it will aid your understanding and arguments.

\vspace{10mm}

Next, read the paper \textit{Simulating Supply} by Bollen and Whaley. Bollen and Whaley take the argument of Culp and Miller at face value and contribute
additional empirical evidence. They employ a parametric Monte Carlo econometric model (though one simpler than Pirrong's) to conduct a counter-factual 
simulation study of what might have happened had MGRM's hedging strategy been left in place for the duration of their committed contracts. They find the following:

\begin{quote}
	\textit{We...calculate what MGR\&M's cash position would have been at the end of March 1997 had the programme been fully funded and left in place. We find that the 
	programme would have earned in excess of $\$1.1$ billion, covering even the most pessimistic accounts of the losses sustained from MGR\&M's programme.}
\end{quote}

\vspace{10mm}

When you have read these papers and made appropriate notes, please answer the following questions.

\vspace{10mm}

\textbf{Question 4.} One of the main points of disagreement between Culp \& Miller and their critics, especially Pirrong, is over the proper hedge ratio for MGRM. 
Had you been the Senior Quantitative Analyst in charge of determining MGRM's strategy, what hedging strategy would have employed? Can you outline a procedure for
determining the appropriate hedging strategy? Explain. 

\vspace{8mm}

\textbf{Note:} Basically, this question is asking you to compare and contrast the neoclassical theory of hedging with Holbrook Working's carrying-charge hedging. Though,
you should feel completely free to add your own novel take on the problem.

\vspace{10mm}

\textbf{Question 5.} Compare Pirrong's empirical evidence with the evidence presented by Bollen and Whaley. Is the evidence in the two papers consistent? If not, why not?
Which one do you find most persuades you? In your answers please refer to the Kennedy and Speed readings, which will help you understand and critique the methodology of 
Bollen and Whaley. 

\vspace{10mm}

\textbf{Question 6.} Energy markets such as the oil market are often historically characterized by backwardation, but have had long periods of contango. Does this matter
for MGRM's synthetic storage strategy? Explain.

\vspace{10mm}

\textbf{Question 7.} At the end of Pirrong's paper, he states \textit{"Given the huge losses incurred in late 1993, a Bayesian estimating the probability distribution of 
MG's information advantage would almost certainly place little weight on the possibility that the firm was well informed, and great weight on the possibility that it did
not possess superior information, regardless of the charitability of his priors concerning the prescience of MG's managers."} Why do you think Pirrong invokes Bayes's Rule
in his concluding statement? (He did not use Bayesian methods to estimate his model) Comment on this statement by Pirrong. Are you convinced by Pirrong's argument? Explain.
In your answer cite the Terrance Speed reading which mentions "not needing to be Bayesian, but only approximately so" (paraphrasing). Bollen and Whaley make no direct reference to Bayes's
Rule. Can their evidence be seen as approximately Bayesian? Compare the Bayesian-ness of the Pirrong and Bollen \& Whaley approaches? (\textit{Hint:} do either of these papers
use a predictive distribution or something akin to it?)

\newpage

\section*{Part III}

\textbf{Question 8.} In this problem you will simulate values for spot prices and basis for heating oil
            and gasoline following the models of Bollen and Whaley in their paper \textit{Simulating Supply} 
			(which is in the readings file). Specifically, the following models will be used:

 \vspace{5mm}
 \textbf{Spot Prices:}
 \begin{itemize}
  \item $\ln{\left(S_{i,t} / S_{i,t-1}\right)} = \alpha_{i} (\beta_{i} - S_{i,t-1}) + \varepsilon_{i,t} \quad
	  \mbox{for} \quad i = 1, 2$
  \item[]
  \item where $i=1$ for heating oil and $i=2$ for gasoline, and $(\varepsilon_{1,t}, \varepsilon_{2,t}) \sim BVN(0,
	  \sigma_{1}^{2}, 0, \sigma_{2}^{2}, \rho)$. And $\rho$ is the correlation coefficient between heating oil and gas spot returns. 
 \end{itemize}

 \vspace{3mm}
 \textbf{Basis and Futures Prices:}
 \begin{itemize}
  \item $b_{i,t} = \alpha_{i} b_{i,t-1} + \beta_{i} S_{i,t-1} + \varepsilon_{i,t} \quad \mbox{for} \quad i =
	  1,2$
  \item[]
  \item where again, $i = 1$ for heating oil and $i = 2$ for gasoline and $(\varepsilon_{1,t}, \varepsilon_{2,t}) \sim
	      BVN(0, \sigma_{1}^{2}, 0, \sigma_{2}^{2}, \rho)$. Again, $\rho$ is the correlation between heating oil and
		  gasoline basis. 
  \item[]
  \item You can then obtain simulated futures prices as $F_{i,t} = S_{i,t}e^{b_{i,t}}$ for $i = 1,2$
 \end{itemize}

 \vspace{5mm}
 With this as background, do the following:

 \begin{itemize}
  \item[(a)] Simulate 60 days of prices for spot and daily settlement prices for futures (via the basis equation).
	         Looking at figure $1$ in the paper use initial values of $\$0.69$ for heating oil and $\$0.80$ for
			 gasoline. Use an initial value for heating oil basis of $-0.02$ and $-0.01$ for gasoline.
			 See Table $2$ in the paper for the other parameter values. 
  \item[(b)] Make time series plots labeling the x-axis as date and the y-axis as dollar prices. Assume a starting date
	         of November 15, 1991. Plot spot prices, basis, and futures prices separately but combining the graphs for 
			 heating oil and gasoline together for each. Clearly label the series in each graph. 
  \item[(c)] For the simulated 60-day period calculate the market-to-market cash flows. Assume an initial margin
	         of $10\%$ and variation margin of $85\%$ of the initial margin. Assume a position of 1 contract for each
			 position.
  \item[(d)] For the simulated 60-day period calculate a minimum-variance hedge ratio for both heating oil and unleaded
	         gasoline.
 \end{itemize}


\end{document}
